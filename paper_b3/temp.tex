\documentclass[12pt]{article}
%\usepackage{caption}
\usepackage{setspace,graphicx,epstopdf,amsmath,amsfonts,amssymb,amsthm}
\usepackage{marginnote,datetime,enumitem,subfigure,rotating,fancyvrb}
\usepackage{hyperref,float}
\usepackage[longnamesfirst]{natbib}
\usepackage{mathtools}
\usepackage{caption}
\usepackage{float}
\usepackage{lscape}
\usepackage{graphicx}
\usepackage{flafter} 
\usepackage{tabularx} 
\usepackage{booktabs}
\usepackage{changepage}
\usepackage{setspace}
\usepackage{placeins}
\usepackage{threeparttable}
\usepackage{ragged2e}
\usepackage[export]{adjustbox}
\usepackage{stmaryrd}
\newcolumntype{Y}{>{\raggedleft\arraybackslash}X}% raggedleft column X
\usdate
\usepackage[a4paper,bindingoffset=0.2in,%
            left=0.8in,right=0.8in,top=1in,bottom=0.8in,%
            footskip=.35in]{geometry}



% SCRIPTS:
% use sp19_dailycimad_betas_10_upd for f1
% plan to use SP19_rm_34 to create all the tables with monthly returns

% Number paragraphs and subparagraphs and include them in TOC
\setcounter{tocdepth}{2}

% JF-specific includes:

\usepackage{indentfirst} % Indent first sentence of a new section.
\usepackage{endnotes}    % Use endnotes instead of footnotes


\begin{document}


\setlist{noitemsep}  % Reduce space between list items (itemize, enumerate, etc.)
\onehalfspacing      % Use 1.5 spacing
% Use endnotes instead of footnotes - redefine \footnote command
\renewcommand{\footnote}{\endnote}  % Endnotes instead of footnotes

\author{\large{Mykola Pinchuk}\thanks{\rm Simon Business School, University of Rochester. Email: Mykola.Pinchuk@ur.rochester.edu. \newline I would like to thank Alan Moreira, Yixin Chen, Shuaiyu Chen, Pingle Wang, Xuyanda Qi, David Swanson, Yushan Zhuang and Robert Mann for helpful comments. All errors are my own.}}

\title{\bf Cross-Industry Dispersion and the Cross-Section of Expected Returns}

\date{29 October 2019 2pm}  
% this is a version after Robert`s proofreading on 9/13/19, 3-4.20 pm.

\maketitle
\thispagestyle{empty}

\bigskip

\normalsize
\vspace{1cm}

\centerline{\bf Abstract}

\vspace{0.5cm}

\begin{onehalfspace}  % Double-space the abstract and don't indent it
  \noindent This paper examines cross-industry dispersion (CID), defined as a mean absolute deviation of returns of 49 industry portfolios. I find that expected stock returns are related cross-sectionally to the sensitivities of returns to innovations in CID. Annualized returns of the stocks with high sensitivity to CID are 8.2\% lower than the returns of the stocks with low sensitivity. Abnormal returns with respect to the best factor model are 5.8\%, suggesting that common factors can not explain this return spread. CID predicts unemployment, consistent with the hypothesis that CID is a proxy for unemployment risk from sectoral shifts. 
\end{onehalfspace}
\medskip


\clearpage
%\doublespacing
\setstretch{1.525}





% Table created by stargazer v.5.2.2 by Marek Hlavac, Harvard University. E-mail: hlavac at fas.harvard.edu
% Date and time: Tue, Oct 29, 2019 - 7:06:00 PM
\begin{table}[!htbp] \centering 
  \caption{Predictive regressions for unemployment} 
  \label{} 
\begin{tabular}{@{\extracolsep{0pt}}lcccccc} 
\\[-1.8ex]\hline 
\hline \\[-1.8ex] 
 & \multicolumn{6}{c}{\textit{Dependent variable:}} \\ 
\cline{2-7} 
 & \multicolumn{2}{c}{Unemployment} & \multicolumn{2}{c}{LT Unemployment} & \multicolumn{2}{c}{ST Unemployment} \\ 
\\[-1.8ex] & (1) & (2) & (3) & (4) & (5) & (6)\\ 
\hline \\[-1.8ex] 
 lcimad\_vw & 2.00$^{**}$ & 1.77$^{**}$ & 1.28$^{**}$ & 1.26$^{**}$ & 0.22 & 0.12 \\ 
  & t = 2.38 & t = 2.30 & t = 2.14 & t = 2.27 & t = 1.22 & t = 0.76 \\ 
  & & & & & & \\ 
 lvwretd &  & 0.06 &  & 0.34 &  & $-$0.15$^{*}$ \\ 
  &  & t = 0.14 &  & t = 1.26 &  & t = $-$1.93 \\ 
  & & & & & & \\ 
 lroll\_sd24 &  & 0.11 &  & 0.07$^{*}$ &  & 0.01 \\ 
  &  & t = 1.53 &  & t = 1.95 &  & t = 0.87 \\ 
  & & & & & & \\ 
 Constant & 0.01 & 0.01 & 0.01 & 0.01 & $-$0.002 & $-$0.002 \\ 
  & t = 0.17 & t = 0.17 & t = 0.28 & t = 0.30 & t = $-$0.19 & t = $-$0.24 \\ 
  & & & & & & \\ 
\hline \\[-1.8ex] 
Observations & 850 & 850 & 850 & 850 & 850 & 850 \\ 
R$^{2}$ & 0.06 & 0.07 & 0.08 & 0.10 & 0.01 & 0.02 \\ 
Adjusted R$^{2}$ & 0.06 & 0.06 & 0.08 & 0.10 & 0.01 & 0.02 \\ 

\hline 
\hline \\[-1.8ex] 
\textit{Note:}  & \multicolumn{6}{r}{$^{*}$p$<$0.1; $^{**}$p$<$0.05; $^{***}$p$<$0.01} \\ 
\end{tabular} 
\end{table}



\end{document}